\section{\textit{Naive Bayes Classifier}}
\subsection{Pengertian \textit{Naive Bayes Classifier}}
\textit{Naive Bayes Classifier} merupakan salah satu metode klasifikasi probabilistik dan statistik yang dikemukakan oleh Thomas Bayes. Algoritma \textit{Naive Bayes} biasa digunakan untuk memprediksi peluang di masa depan dengan berdasarkan pengalaman di masa sebelumnya, biasanya disebut dengan Teorema \textit{Bayes}. 
\textit{Naive Bayes} didasarkan pada asumsi penyederhanaan bahwa nilai atribut secara konditional saling bebas jika diberikan 

























\section{Algoritma Apriori}
\subsection{Pengertian Algoritma Apriori}
Algoritma apriori merupakan jenis aturan asosiasi pada data mining, dimana aturan yang menyatakan asosiasi antara atribut-atribut yang disebut \textit{market basket analysis} \cite{ye2006parallel}. \textit{Asosiation rule} adalah teknik data minig yang digunakan untuk menentukan atura asosiatif antar suatu kombinasi \textit{item}. Algoritma apriori yang bertujuan untuk menemukan \textit{frequent itemsets} pada sekumpulan data yang memenuhi syarat minimum \textit{support} dan minimum \textit{confidence} yang telah ditentukan.
\par
Algoritma Apriori menggunakan pengetahuan frekuensi atribut yang  sebelumnya telah didapat untuk dilakukan proses informasi selanjutnya. Pada algoritma Apriori  menentukan kandidat yang mungkin muncul dengan cara memperhatikan nilai minimum \textit{support} dan nilai minimum \textit{confidence}. \textit{Support} adalah nilai pengunjung atau persentase kombinasi sebuah \textit{item} dalam \textit{database}. Sedangkan \textit{confidence} adalah nilai kepastian yang menentukan kuatnya hubungan antar \textit{item} dalam sebuah Apriori. \textit{Confidence} dapat dicari setelah pola frekuensi munculnya sebuah \textit{item} ditemukan.


\subsection{Kelebihan dan Kelemahan Algoritma Apriori}
\par Beberapa kelebihan yang dimiliki oleh algoritma apriori adalah sebagai berikut \cite{fauzy2016penerapan}:
 \begin{enumerate}
\item Menghasilkan kombinasi yang sangat banyak sehingga sangat tidak efisien. 
\item Lebih sederhana
\item Dapat menangani data yang besar
\item Apriori memiliki akurasi rules yang tinggi
\end{enumerate}

\par Beberapa kelemahan yang dimiliki oleh algoritma apriori adalah sebagai berikut:
 \begin{enumerate}
\item Proses \textit{scan database} yang dilakukan setiap kali iterasi, sehingga waktu yang diperlukan bertambah dengan makin banyak iterasi
\item Proses \textit{scanning} yang
dilakukan pada apriori yang berulang kali membuat tingkat kecepatan menjadi lambat

\end{enumerate}

\subsection{Tahap-Tahap Menghitung Algoritma Apriori}
Tahap-tahap yang dilakukan untuk mendapatkan \textit{frequent itemset} dalam algoritma apriori adalah sebagai berikut \cite{triyanto2014association}:
 \begin{enumerate}
\item Penggabungan (\textit{Join})
\par Proses penggabungan dilakukan dengan melakukan kombinasi terhadap satu \textit{item} dengan \textit{item} yang lainnya hingga tidak dapat terbentuk kombinasi lagi.
\item Pemangkasan (Prune)
\par proses pemangkasan merupakan hasil dari \textit{item} yang sudah dikombinasi kemudian dilakukan pemangkasan dengan menggunakan minimum \textit{support} yang telah ditentukan sebelumnya. 
\end{enumerate}

\par Prinsip-prinsip dalam algoritma apriori adalah sebagai berikut \cite{tampubolon2013implementasi}:
 \begin{enumerate}
\item Mengumpulkan \textit{item} tunggal kemudian mencari \textit{item} dengan nilai tersebar.
\item Mendaptkan \textit{candidate pairs} kemudian menghitung \textit{large pairs} dari setiap \textit{item} yang ada
\item Menemukan \textit{candidate triplets} dari setiap \textit{item} dan seterusnya
\item Setiap \textit{subset} dari sebuah \textit{frequent itemset} harus menjadi \textit{frequent}.
\end{enumerate}

\pagebreak
\par Algoritma apriori terbagi menjadi beberapa tahap yaitu iterasi (\textit{pass}). Setiap iterasi mnghasilkan sebuah pola frekuensi tinggi dengan panjang yang sama dimulai dari iterasi pertama yang menghasilkan pola frekuensi tinggi dengan panjang satu. Pada iterasi pertama, nilai \textit{support} dari setiap \textit{item} didapat, \textit{item} yang memiliki nilai \textit{support} lebih besar dari nilai minimum \textit{support} akan diambil sebagai pola frekuensi tinggi dengan panjang satu atau 1-\textit{itemset}. 1-i artinya satu set yang terdiri dari k \textit{item}. Kemuidan untuk kandidat 2-\textit{itemset} akan dihitung nilai \textit{support}-nya dengan melakukan \textit{scan} terhadap \textit{database}.
\par
\textit{Support} yang diamksudkan disini artinya jumlah transaksi pada \textit{database} yang mengandung keduan \textit{item} dalam kandidat 2-\textit{itemset}. Setelah nilai \textit{support} dari semua kandidat 2-\textit{itemset} didapat, kandidat 2-\textit{itemset} yang memenuhi syarat nilai minimum \textit{support} dapat ditetapkan sebagai 2-\textit{itemset} yang merupakan pola frekuensi tinggi dengan panjang 2 \cite{nursikuwagus2016implementasi}.
\par
Untuk iterasi ke-k dapat dibagi menjadi beberapa bagian:
 \begin{enumerate}
\item Pembentukan kandidat \textit{itemset}.
\par Kandidat k-\textit{itemset} dibentuk dari kombinasi (k-1)-\textit{itemset} yang didapat dari iterasi sebelumnya. Karakteristik dari algoritma apriori adalah adanya pemangkaan terhadap kandidat k-iterasi yang subsetnya berisikan k-1 \textit{item} tidak termasuk kedalam pola frekuensi tinggi dengan panjang k-1.
\item Perhitungan nilai \textit{support} dari setiap kandidat k-\textit{itemset}.
\par Nilai \textit{support} dari setiap kandidat k-\textit{itemset} didapt dengan melakukan \textit{scan} terhadap \textit{database} untuk menghitung jumlah transaksi yang memuat seluruh \textit{item} didalam kandidat k-\textit{itemset} tersebut. Ini merupakan karakteristik dari algoritma ini dimana diperlukan perhitungn dengan melakukan \textit{scan} seluruh \textit{database} sebanyak k-\textit{itemset} terpanjang.
\item Menetapkan pola frekuensi tinggi.

\par 
Pola frekuensi tinggi yang memuat k \textit{item} atau k-\textit{itemset} ditetapkan dari kandidat k-\textit{itemset} yang nilai \textit{support}-nya lebih besar dari nilai minimum \textit{support} yang telah ditetapkan sebelumnya.
\item Apabila tidak mendapatkan pola frekuensi tinggi yang baru, maka semua proses berhenti. Bila tidak, makan k ditambah satu dan kembali ke bagian awal.
\end{enumerate}
.
\\
\\
\\
\\
\\
\\
\subsection{Rumus Algoritma Apriori}
\par Algoritma apriori menggunakan pengetahuan frekuensi atribut yang telah diketahui untuk selanjut dilakukan proses informai selanjutnya. Algoritma apriori melakukan penentuan kandidat yang mungkin muncul dengan cara memerhatikan nilai  minimum \textit{support} dan \textit{confidence} yang telah ditentukan sebelumnya \cite{yanto2015implementasi}.

\par 
\textbf{\textit{Support}}: sebuah ukuran yang menunjukkan besarnya tingkat dominasi sebuah \textit{item} atau \textit{itemset} dari keseluruhan transaksi. Ukuran ini akan menentukan kelayakan sebuah \textit{itemset} atau \textit{item} untuk dihitung nilai \textit{confidence} tersebut dapat digunakan untuk menentukan tingkat dminasi \textit{item} tunggal. Misalnya, semua tranaksi yang tersedia, berapa besar tingkat dominasi yang menunjukkan bahwa \textit{item} A \& B dibeli secara bersamaan.
\par
\textbf{\textit{Confidence}}: sebuahukuran yang menunjukkan hubungan antara kedua \textit{itemset} secara \textit{conditional}. Misalnya, seberapa sering \textit{item} B dibeli jika pelanggan membeli \textit{item} A.

\par
Nilai \textit{support} merupakan sebuah nilai presentase dari kombinasi sebuah \textit{item} didalam \textit{database}.
Rumusnya:
\begin{equation}
    Support A =\frac{Jumlah Transaksi Mengandung A}{Jumlah Transaksi} 
\end{equation}
\par
Sedangkan, Nilai \textit{confidence} adalah sebuah nilai yang pasti yaitu nilai kuatnya suatu hubungan antara \textit{item-item} dalam sebuah apriori. \textit{Confidence} didapat setelah pola frekuensi munculnya item telah ditemukan. Rumus untuk menghitung \textit{confidence}:
\par
Contohnya ditentukan aturan A - B maka:
 \begin{equation}
    Support (A,B) =\frac{Jumlah Transaksi Mengandung A dan B}{Jumlah Transaksi} 
\end{equation}
.
\\
\\
\\
\\
\\
\\
\\
\\
\\
\\
\\
\\
\\
\\
\subsection{Contoh Soal Algoritma Apriori}
Diketahui telah terjadi transaksi permintaan pengadan barang sebanyak 12 kali, dan jenis item yang dibeli adalah PC, CPU, Speaker, Wifi, CCTV, AC dan Proyektor. Tentukan seberapa sering sebuah \textit{project} meminta produk untuk diadakan pada Divisi \textit{Procurement} dalam penentuan pola pengadaan barang.Untuk penyelesaiannya adalah sebagai berikut:

\subsubsection{Data Transaksi Permintaan Pengadaan Barang}
Berdasarkan data transaksi divisi \textit{Procurement} PT. Cinovasi Rekaprima pada periode Januari dan Desember 2018 dilakukan akumulasi data transaksi permintaan pengadaan barang yang di rekap berdasarkan \textit{quotation} dapat dilihat pada Tabel VI.1 sebagai berikut :
\begin{table}[!h]
\caption{Pola Transaksi Permintaan Pengadaan Barang}
\begin{center}
\begin{tabular}{|l|l|l|}
\hline
No & \begin{tabular}[c]{@{}l@{}}Nama Project\end{tabular} & Nama Produk                 \\ \hline
1  & P1                                                      & PC,CPU,CCTV,AC              \\ \hline
2  & P2                                                      & PC,CPU                      \\ \hline
3  & P3                                                      & PC,CPU,Speaker,Wifi,CCTV,   \\ \hline
4  & P4                                                      & CCTV,AC,Proyektor           \\ \hline
5  & P5                                                      & PC,CPU,Speaker,Wifi         \\ \hline
6  & P6                                                      & Speaker,Wifi                \\ \hline
7  & P7                                                      & PC,CPU,TV,CCTV,             \\ \hline
8  & P8                                                      & PC,Speaker,Wifi,Proyektor   \\ \hline
9  & P9                                                      & Speaker,Wifi,CCTV,Proyektor \\ \hline
10 & P10                                                     & PC,CPU,Speaker,Wifi         \\ \hline
11 & P11                                                     & PC,TV                       \\ \hline
12 & P12                                                     & PC,CPU,AC                   \\ \hline
\end{tabular}
\end{center}
\end{table}

\pagebreak
\subsubsection{Tabulasi Data Transaksi}
Pada data transaksi permintaan pengadaan barang di bentuk tabel tabular yang akan mempermudah dalam mengetahui berapa banyak \textit{item} yang ada diminta dalam setiap transaksi seperti pada Tabel VI.2 berikut:
\begin{table}[!h]
\caption{Pola Transaksi Permintaan Pengadaan Barang}
\begin{center}
\begin{tabular}{|l|l|l|l|l|l|l|l|l|}
\hline
             & \multicolumn{8}{l|}{Nama Produk}                       \\ \hline
Nama Project & PC & CPU & Speaker & Wifi & TV & CCTV & AC & Proyektor \\ \hline
P1           & 1  & 1   & 0       & 0    & 0  & 1    & 1  & 0         \\ \hline
P2           & 1  & 1   & 1       & 1    & 0  & 1    & 0  & 0         \\ \hline
P3           & 1  & 1   & 1       & 0    & 1  & 0    & 0  & 0         \\ \hline
P4           & 0  & 0   & 0       & 0    & 0  & 1    & 1  & 1         \\ \hline
P5           & 1  & 1   & 1       & 1    & 0  & 0    & 0  & 0         \\ \hline
P6           & 0  & 0   & 1       & 1    & 0  & 0    & 0  & 0         \\ \hline
P7           & 1  & 1   & 0       & 0    & 1  & 1    & 0  & 0         \\ \hline
P8           & 1  & 0   & 1       & 1    & 0  & 0    & 0  & 1         \\ \hline
P9           & 0  & 0   & 1       & 1    & 0  & 1    & 0  & 1         \\ \hline
P10          & 1  & 1   & 1       & 1    & 0  & 0    & 0  & 0         \\ \hline
P11          & 1  & 0   & 0       & 0    & 1  & 0    & 0  & 0         \\ \hline
P12          & 1  & 1   & 0       & 0    & 0  & 0    & 1  & 0         \\ \hline
jumlah       & 9  & 7   & 7       & 6    & 3  & 5    & 3  & 3         \\ \hline
\end{tabular}
\end{center}
\end{table}

\begin{table}[!h]
\caption{Tabel Inisialisasi Data Barang}
\begin{center}
\begin{tabular}{|l|l|l|l|}
\hline
Produk  & Inisial & Produk    & Inisial \\ \hline
PC      & K       & TV        & S       \\ \hline
CPU     & M       & CCTV      & A       \\ \hline
Speaker & R       & AC        & T       \\ \hline
Wifi    & L       & Proyektor & G       \\ \hline
\end{tabular}
\end{center}
\end{table}

\pagebreak
\subsubsection{Pembentukan \textit{Itemset}}
\par Dalam perhitungan algoritma apriori untuk menentukan pola pengadaan barang ini ditentukan  nilai minimum \textit{support} dan minimum \textit{confidence}. Jumlah minimum \textit{support} adalah 40\% dan minimum \textit{confidence} adalah 60\%.
Nilai minimum \textit{support} yang terlalu besar menyebabkan jumlah kombinasi \textit{itemset} yang akan memenuhi minimum \textit{support} sedikit bahkan tidak \textit{item} yang mencapai minimum \textit{support}.
\par
Pembentukan confidance pun tidak dapat dihitung karena tidak ada \textit{item} yang memenuhi minimum \textit{support} dan menyebabkan proses perhitungan selesai pada pembentukan \textit{support}. Oleh karena itu penentuan nilai minimum \textit{support} mengambil nilai tengah dimana minimum \textit{support}  40\% dan minimum \textit{confidence} 60\% dari total 100\% agar \textit{itemset} yang dikombinasi dapat memenuhi nilai minimum.
Nilai minimum \textit{confidance} lebih besar dibandignkan minimum \textit{support}, karena dalam perhitungan \textit{confidance} akan menyaring \textit{itemset} teretntu untuk dilakukan pembentukan aturan asosiasi.

\begin{enumerate}
\item Kombinasi 1 \textit{Itemset}
\par Berikut ini merupakan proses pembentukan C1 atau disebut dengan kombinasi 1 \textit{itemset} dengan jumlah minimum \textit{support} = 40\%. Maka, hasil pembentukan kombinasi 1 \textit{itemset} dengan  adalah sebagai berikut :

\begin{equation}
    Support A =\frac{Jumlah Transaksi Mengandung A}{Jumlah Transaksi} 
    \end{equation}

\begin{table}[!h]
\caption{Minimum Support dari 1 Itemset}
\begin{center}
\begin{tabular}{|l|l|l|l|}
\hline
1 itemset & Jumlah & support & support (\%) \\ \hline
K         & 9      & 0.75    & 75\%         \\ \hline
M         & 7      & 0.58    & 58\%         \\ \hline
R         & 7      & 0.58    & 58\%         \\ \hline
L         & 6      & 0.5     & 50\%         \\ \hline
S         & 3      & 0.25    & 25\%         \\ \hline
A         & 5      & 0.42    & 42\%         \\ \hline
T         & 3      & 0.25    & 25\%         \\ \hline
G         & 3      & 0.25    & 25\%         \\ \hline
\end{tabular}
\end{center}
\end{table}

\par Dari proses pembentukan \textit{itemset} dengan minimum \textit{support} 40\% dapat diketahui yang memenuhi standar minimum \textit{support} yaitu pada PC dan CPU dan Speaker, Wifi. Kemudian dari hasil pembentukan 1 \textit{itemset} akan dilakukan kombinasi 2 \textit{itemset}.

\pagebreak
\item Kombinasi  2 \textit{Itemset}
\par Berikut ini merupakan pembentukan 2 itemset berdasarkan data kombinasi 1 \textit{itemset}  yang memenuhi standar minimum \textit{support} yang sudah disediakan pada Tabel VI.4 diatas. Proses pembentukan C2 atau disebut dengan 2 \textit{itemset} dengan jumlah minimum \textit{support} = 40\% Dapat diselesaikan dengan rumus berikut:

\begin{equation}
Support (A,B) =\frac{Jumlah Transaksi Mengandung A dan B}{Jumlah Transaksi} 
\end{equation} 

\begin{table}[!ht]
\caption{Support Kombinasi 2 Itemset}
\centering
\begin{tabular}{|l|l|l|l|}
\hline
2 itemset & jumlah & support & support (\%) \\ \hline
K,M       & 7      & 0.58    & 58\%         \\ \hline
K,R       & 4      & 0.33    & 33\%         \\ \hline
K,L       & 3      & 0.25    & 25\%         \\ \hline
K,A       & 2      & 0.17    & 17\%         \\ \hline
M,R       & 4      & 0.33    & 33\%         \\ \hline
M,L       & 3      & 0.25    & 25\%         \\ \hline
M,A       & 3      & 0.25    & 25\%         \\ \hline
R,L       & 6      & 0.50    & 50\%         \\ \hline
R,A       & 2      & 0.17    & 17\%         \\ \hline
L,A       & 2      & 0.17    & 17\%         \\ \hline
\end{tabular}
\end{table}

\par Dari proses pembentukan \textit{itemset} pada dengan minimum \textit{support} 40\% dapat diketahui yang memenuhi standar minimum \textit{support} yaitu pada  PC, CPU, Speaker, Wifi, dan CCTV. Kemudian dari hasil pembentukan 2 \textit{itemset} akan dilakukan kombinasi 3 \textit{itemset}.

\pagebreak
\item Kombinasi 3 \textit{Itemset}
\par Berikut ini merupakan pembentukan  \textit{itemset} berdasarkan data kombinasi 2 \textit{itemset}  yang memenuhi standar minimum \textit{support} yang sudah disediakan pada Tabel VI.5 diatas. Proses pembentukan C3 atau disebut dengan 3 \textit{itemset} dengan jumlah minimum \textit{support} = 40\% Dapat diselesaikan dengan rumus berikut:
\begin{equation}
Support (A,B,C) =\frac{Jumlah Transaksi Mengandung A,B dan C}{Jumlah Transaksi} 
\end{equation}
\begin{table}[!ht]
\caption{Support Kombinasi 3 Itemset}
\centering
\begin{tabular}{|l|l|l|l|}
\hline
3 itemset & Jumlah & support & support (\%) \\ \hline
KMR       & 4      & 0.33    & 33\%         \\ \hline
KML       & 2      & 0.17    & 17\%         \\ \hline
MRL       & 3      & 0.25    & 25\%         \\ \hline
KRL       & 3      & 0.25    & 25\%         \\ \hline
\end{tabular}
\end{table}

\par Dari kombinasi 3 \textit{itemset} dengan minimum \textit{support} 40\% dapat diketahui kombinasi 3 \textit{itemset} yang memenuhi standar minimum \textit{support} adalah Wifi, Proyektor, Speaker, AC dan CCTV dengan nilai \textit{support} sebesar 42\% . Karena Kombinasi 3 \textit{itemset} tidak ada yang memenuhi minimal \textit{support} 40\%, maka kombinasi 3 \textit{itemset} yang memenuhi untuk pembentukan asosiasi.
\end{enumerate}

\subsection{Pembentukan Confidance}
\par Setelah semua pola frekuensi tinggi ditemukan, kemudian dicari aturan asosiasi yang memenuhi syarat minimum untuk \textit{confidence} dengan menghitung \textit{confidence} aturan asosiatif A-B . 
\par Minimum \textit{Confidence} = 60\% 
\par Nilai \textit{Confidence} dari aturan A-B diperoleh :
\begin{equation}
Confidence = P(B|A) =\frac{Jumlah Transaksi Mengandung A dan B}{Jumlah Transaksi Mengandung A} 
\end{equation}
\par Pada tabel berikut menunjukan itemset 2 yang telah dihitung nilai \textit{confidence} dan telah diseleksi oleh minimal \textit{confidence} adalah 60\%.
\begin{table}[!ht]
\caption{Support Kombinasi 3 Itemset}
\centering
\begin{tabular}{|l|l|l|l|}
\hline
2Itemset & jumlah & confidance & confidance (\%) \\ \hline
K,M      & 5/9    & 0.78       & 78\%            \\ \hline
R,L      & 5/7    & 0.86       & 86\%            \\ \hline
\end{tabular}
\end{table}

\par Pada tabel berikut menunjukan \textit{itemset} 3 yang telah dihitung nilai \textit{confidence} dan telah diseleksi oleh minimal \textit{confidence} 60\%.
\begin{table}[!ht]
\caption{Confidance Kombinasi 3 Itemset}
\centering
\begin{tabular}{|l|l|l|l|}
\hline
3Itemset & jumlah & confidance & confidance (\%) \\ \hline
KMR      & 4      & 0.44       & 44\%            \\ \hline
KML      & 3      & 0.33       & 33\%            \\ \hline
RLK      & 4      & 0.57       & 57\%            \\ \hline
RLM      & 3      & 0.43       & 43\%            \\ \hline
\end{tabular}
\end{table}

\subsection{Pembentukan Aturan Asosiasi}
\par Berikut ini merupakan tabel \textit{association rule}, dimana \textit{support} dan \textit{confidence} dari masing – masing \textit{itemset} dikalikan agar dapat mengetahui mana \textit{association rule} yang paling besar nilanya. Dikarenakan Batasan \textit{final association rule} yang ditentukan hanya 2, maka aturan yang diambil hanya 2 aturan tertinggi seperti yang ada pada tbael berikut:
\begin{table}[!ht]
\caption{Pembentukan Aturan Asosiasi Kombinasi 2 Itemset}
\centering
\begin{tabular}{|l|l|l|l|}
\hline
2 itemset                                       & support & confidance & sxc\\ \hline
Jika membeli Kursi, maka akan membeli Meja      & 58\%    & 78\%       & 45.4\%               \\ \hline
Jika membeli Rak Buku, maka akan membeli Lemari & 50\%    & 86\%       & 42.9\%               \\ \hline
\end{tabular}
\end{table}

\par Jadi, hadil dari pembentukan aturan asosiasi ini akan diketahui pola pengadaan barang dari transaksi permintaan pengadaan barang. Hal ini dapat dijadi sebagai rekomendasi produk dalam proses pengaan barang

\subsection{Referensi}
